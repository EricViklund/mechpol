\documentclass{beamer}
\mode<presentation>
\usepackage[orientation=landscape,size=a0,scale=1.4]{beamerposter}
\usetheme{Warsaw}
\usepackage{siunitx}
\title{Improving Nb\textsubscript{3}Sn Cavity Performance Using Mechanical Polishing}%
\author[shortname]{\large Eric Viklund \inst{1, 2} \and David Burk \inst{2} \and David N. Seidman \inst{1} \and Sam Posen \inst{2}}
\institute[shortinst]{\large \inst{1} Department of Materials Science and Engineering, Northwestern University \newline \inst{2} Fermi National Accelerator Laboratory}
\date{\today}%
\begin{document}%
    \begin{frame}{}
        \maketitle
        \begin{abstract}%
            In this study we will show a new method of polishing for Nb\textsubscript{3}Sn cavities known as centrifugal barrel polishing (CBP). Using this method, Nb\textsubscript{3}Sn coated samples are polished to a surface roughness comparable to a traditional Nb cavity after electropolishing (EP). We also investigate different methods of cleaning the Nb\textsubscript{3}Sn surface after CBP to remove residual abrasive particles. The polished Nb\textsubscript{3}Sn surface is analyzed using confocal laser microscopy, and scanning electron microscopy (SEM) is used to image the surface and measure the surface roughness after polishing. Transmission electron microscopy (TEM) is also used for high resolution analysis of the surface after polishing. Finally, we report on the performance of a barrel polished Nb\textsubscript{3}Sn cavity.
            %
        \end{abstract}%

        \begin{columns}[t]
            \begin{column}{0.32\linewidth}
                \begin{block}{\label{sec:introduction}Introduction}
                    Superconducting radiofrequency (SRF) cavities are essential components in particle accelerators used for various scientific and industrial applications. These cavities can achieve higher accelerating gradients and lower surface resistance than traditional, normal-conducting cavities. SRF cavities coated with a layer of Nb\textsubscript{3}Sn can reach up to 100~MV/m of accelerating field and have a promising future in high-energy linacs or small-scale industrial accelerators. However, surface roughness is a limiting factor for Nb\textsubscript{3}Sn SRF cavity performance. The centrifugal barrel polishing (CBP) method is a promising technique to mechanically polish SRF cavities. In this paper, the authors propose a procedure to mechanically polish Nb\textsubscript{3}Sn cavities using CBP to improve their RF performance. They first polish Nb\textsubscript{3}Sn coated samples to determine the optimum parameters and then apply the technique to a Nb\textsubscript{3}Sn coated, 1.3~GHz, TESLA geometry SRF cavity. The RF performance of the cavity was tested before and after the CBP treatment.
                \end{block}
                \begin{block}{\label{sec:backgroundinformation}Background Information}
                    \begin{itemize}
                        \item Centrifugal Barrel Polishing (CBP) is a technique used to polish niobium cavities without using toxic chemicals such as HF.
                        \item CBP uses a custom built tumbling machine that can fit up to 9-cell size cavities, and abrasive slurry to accelerate the polishing media against the cavity surface with up to 6g of force.
                    \end{itemize}  
                \end{block}
            \end{column}
            \begin{column}{0.32\linewidth}    
                \begin{block}{\label{sec:samplestudy}Sample Study}
                    Since Nb\textsubscript{3}Sn is a relatively unexplored material, there are no established polishing parameters or abrasive materials to achieve a good surface finish. To allow for rapid iteration and microscopy surface analysis, we first perform polishing experiments on Nb\textsubscript{3}Sn samples. To evaluate the performance of CBP, the surface roughness of the polished samples is measured using con-focal laser microscopy and the surface is analyzed using scanning and transmission electron microscopy (SEM and TEM). The material removal rate is measured using focused ion-beam tomography.
                    \begin{itemize}                        
                        \item A coupon cavity is used to test the centrifugal barrel polishing method on Nb\textsubscript{3}Sn samples in a realistic environment. The samples sit flush with the inside surface of the coupon cavity, where they experience identical polishing conditions to a real cavity surface.
                        \item The Nb\textsubscript{3}Sn samples and Nb\textsubscript{3}Sn cavity used in this study were coated at Fermilab in a high-vacuum furnace.
                        \item The surface roughness of the polished samples is measured using confocal laser microscopy, and the surface is analyzed using scanning and transmission electron microscopy (SEM and TEM). 
                        \item The removal rate of different abrasive materials has been measured using focused ion-beam tomography.
                        \item The smoothness of the samples improves as longer polishing is applied. Material is preferentially removed from the highest point on the surface, causing the sharp peaks on the surface to be removed quickly while valleys in the surface are left untouched. This is different from chemical polishing methods, which preferentially smooth areas with high curvature.
                        \item After 6~hours of polishing, the surface roughness is comparable to the surface roughness of the well-performing, thinly coated Nb\textsubscript{3}Sn coatings created at Fermilab. After 8~hours of polishing, the surface roughness is comparable to a typical niobium surface after EP.
                    \end{itemize}
                \end{block}
            \end{column}
            \begin{column}{0.32\textwidth}
                \begin{block}{\label{sec:cavitycbp}Polishing a Nb\textsubscript{3}Sn Cavity Using CBP}
                    Given that CBP was able to produce a smooth surface on Nb\textsubscript{3}Sn samples, the next step is to apply the polishing to a Nb\textsubscript{3}Sn cavity. A Nb\textsubscript{3}Sn-coated cavity was polished using the felt cube polishing media with a 50~nm alumina abrasive particle suspension, chosen to avoid the risk of silicon contamination in the coating furnace. The cavity was polished for 4~hours followed by high-pressure water rinsing and ultrasonic cleaning for 30~minutes to remove any residual abrasive material left by the polishing process. These parameters were chosen as a conservative estimate to minimize the possibility of removing the Nb\textsubscript{3}Sn film and allow for more material removal in the future while still providing a considerable improvement in surface roughness.
                    \begin{itemize}
                        \item The Nb\textsubscript{3}Sn-coated cavity was polished using CBP leading to a shiny surface.
                        \item A re-coating procedure was applied to repair surface damage and subsurface defects at 1,000°C, using one third of the normal amount of tin and no SnCl\textsubscript{2}.
                        \item RF performance of the cavity was tested three times using the vertical test stand (VTS) at FNAL.
                        \item The as-coated performance was poor, with a maximum gradient of around 10MV/m and Q of 10\textsuperscript{10} at 4.4K.
                        \item After polishing, the cavity exhited Q-slope and the maximum gradient was only 5MV/m.
                        \item After the re-coating procedure, the Q-slope was ameliorated, and the maximum accelerating gradient increased to 15MV/m.
                        \item The quality factor of the cavity was also improved over the as-coated state at 2.0K, but not at 4.4~K.
                    \end{itemize}
                \end{block}
            \end{column}
        \end{columns}
    \end{frame}
\end{document}